% Options for packages loaded elsewhere
\PassOptionsToPackage{unicode}{hyperref}
\PassOptionsToPackage{hyphens}{url}
\documentclass[
]{article}
\usepackage{xcolor}
\usepackage[margin=2.5cm]{geometry}
\usepackage{amsmath,amssymb}
\setcounter{secnumdepth}{5}
\usepackage{iftex}
\ifPDFTeX
  \usepackage[T1]{fontenc}
  \usepackage[utf8]{inputenc}
  \usepackage{textcomp} % provide euro and other symbols
\else % if luatex or xetex
  \usepackage{unicode-math} % this also loads fontspec
  \defaultfontfeatures{Scale=MatchLowercase}
  \defaultfontfeatures[\rmfamily]{Ligatures=TeX,Scale=1}
\fi
\usepackage{lmodern}
\ifPDFTeX\else
  % xetex/luatex font selection
\fi
% Use upquote if available, for straight quotes in verbatim environments
\IfFileExists{upquote.sty}{\usepackage{upquote}}{}
\IfFileExists{microtype.sty}{% use microtype if available
  \usepackage[]{microtype}
  \UseMicrotypeSet[protrusion]{basicmath} % disable protrusion for tt fonts
}{}
\makeatletter
\@ifundefined{KOMAClassName}{% if non-KOMA class
  \IfFileExists{parskip.sty}{%
    \usepackage{parskip}
  }{% else
    \setlength{\parindent}{0pt}
    \setlength{\parskip}{6pt plus 2pt minus 1pt}}
}{% if KOMA class
  \KOMAoptions{parskip=half}}
\makeatother
\usepackage{color}
\usepackage{fancyvrb}
\newcommand{\VerbBar}{|}
\newcommand{\VERB}{\Verb[commandchars=\\\{\}]}
\DefineVerbatimEnvironment{Highlighting}{Verbatim}{commandchars=\\\{\}}
% Add ',fontsize=\small' for more characters per line
\usepackage{framed}
\definecolor{shadecolor}{RGB}{248,248,248}
\newenvironment{Shaded}{\begin{snugshade}}{\end{snugshade}}
\newcommand{\AlertTok}[1]{\textcolor[rgb]{0.94,0.16,0.16}{#1}}
\newcommand{\AnnotationTok}[1]{\textcolor[rgb]{0.56,0.35,0.01}{\textbf{\textit{#1}}}}
\newcommand{\AttributeTok}[1]{\textcolor[rgb]{0.13,0.29,0.53}{#1}}
\newcommand{\BaseNTok}[1]{\textcolor[rgb]{0.00,0.00,0.81}{#1}}
\newcommand{\BuiltInTok}[1]{#1}
\newcommand{\CharTok}[1]{\textcolor[rgb]{0.31,0.60,0.02}{#1}}
\newcommand{\CommentTok}[1]{\textcolor[rgb]{0.56,0.35,0.01}{\textit{#1}}}
\newcommand{\CommentVarTok}[1]{\textcolor[rgb]{0.56,0.35,0.01}{\textbf{\textit{#1}}}}
\newcommand{\ConstantTok}[1]{\textcolor[rgb]{0.56,0.35,0.01}{#1}}
\newcommand{\ControlFlowTok}[1]{\textcolor[rgb]{0.13,0.29,0.53}{\textbf{#1}}}
\newcommand{\DataTypeTok}[1]{\textcolor[rgb]{0.13,0.29,0.53}{#1}}
\newcommand{\DecValTok}[1]{\textcolor[rgb]{0.00,0.00,0.81}{#1}}
\newcommand{\DocumentationTok}[1]{\textcolor[rgb]{0.56,0.35,0.01}{\textbf{\textit{#1}}}}
\newcommand{\ErrorTok}[1]{\textcolor[rgb]{0.64,0.00,0.00}{\textbf{#1}}}
\newcommand{\ExtensionTok}[1]{#1}
\newcommand{\FloatTok}[1]{\textcolor[rgb]{0.00,0.00,0.81}{#1}}
\newcommand{\FunctionTok}[1]{\textcolor[rgb]{0.13,0.29,0.53}{\textbf{#1}}}
\newcommand{\ImportTok}[1]{#1}
\newcommand{\InformationTok}[1]{\textcolor[rgb]{0.56,0.35,0.01}{\textbf{\textit{#1}}}}
\newcommand{\KeywordTok}[1]{\textcolor[rgb]{0.13,0.29,0.53}{\textbf{#1}}}
\newcommand{\NormalTok}[1]{#1}
\newcommand{\OperatorTok}[1]{\textcolor[rgb]{0.81,0.36,0.00}{\textbf{#1}}}
\newcommand{\OtherTok}[1]{\textcolor[rgb]{0.56,0.35,0.01}{#1}}
\newcommand{\PreprocessorTok}[1]{\textcolor[rgb]{0.56,0.35,0.01}{\textit{#1}}}
\newcommand{\RegionMarkerTok}[1]{#1}
\newcommand{\SpecialCharTok}[1]{\textcolor[rgb]{0.81,0.36,0.00}{\textbf{#1}}}
\newcommand{\SpecialStringTok}[1]{\textcolor[rgb]{0.31,0.60,0.02}{#1}}
\newcommand{\StringTok}[1]{\textcolor[rgb]{0.31,0.60,0.02}{#1}}
\newcommand{\VariableTok}[1]{\textcolor[rgb]{0.00,0.00,0.00}{#1}}
\newcommand{\VerbatimStringTok}[1]{\textcolor[rgb]{0.31,0.60,0.02}{#1}}
\newcommand{\WarningTok}[1]{\textcolor[rgb]{0.56,0.35,0.01}{\textbf{\textit{#1}}}}
\usepackage{graphicx}
\makeatletter
\newsavebox\pandoc@box
\newcommand*\pandocbounded[1]{% scales image to fit in text height/width
  \sbox\pandoc@box{#1}%
  \Gscale@div\@tempa{\textheight}{\dimexpr\ht\pandoc@box+\dp\pandoc@box\relax}%
  \Gscale@div\@tempb{\linewidth}{\wd\pandoc@box}%
  \ifdim\@tempb\p@<\@tempa\p@\let\@tempa\@tempb\fi% select the smaller of both
  \ifdim\@tempa\p@<\p@\scalebox{\@tempa}{\usebox\pandoc@box}%
  \else\usebox{\pandoc@box}%
  \fi%
}
% Set default figure placement to htbp
\def\fps@figure{htbp}
\makeatother
\setlength{\emergencystretch}{3em} % prevent overfull lines
\providecommand{\tightlist}{%
  \setlength{\itemsep}{0pt}\setlength{\parskip}{0pt}}
\usepackage{xcolor}
\usepackage{titlesec}

\definecolor{mainblue}{HTML}{1F4E79}
\definecolor{subblue}{HTML}{3C78D8}
\definecolor{lightgray}{gray}{0.30}

% Niveau 1 (#) : grand + trait sous le titre (pas de #1)
\titleformat{\section}{\normalfont\bfseries\color{mainblue}\huge}{\thesection}{1em}{}[\vspace{0.3em}\titlerule]
\titlespacing*{\section}{0pt}{1.8em plus .5em minus .5em}{1em}

% Niveau 2 (##) : grand + trait (pas de #1)
\titleformat{\subsection}{\normalfont\bfseries\color{subblue}\Large}{\thesubsection}{1em}{}[\vspace{0.2em}\titlerule]
\titlespacing*{\subsection}{0pt}{1.4em plus .3em minus .3em}{0.8em}

% Niveau 3 (###)
\titleformat{\subsubsection}{\normalfont\bfseries\color{black}\large}{\thesubsubsection}{1em}{}
\titlespacing*{\subsubsection}{0pt}{1em plus .3em minus .3em}{0.6em}

% Niveau 4 (####)
\titleformat{\paragraph}{\normalfont\bfseries\normalsize\color{lightgray}}{\theparagraph}{1em}{}
\titlespacing*{\paragraph}{0pt}{0.8em plus .2em minus .2em}{0.4em}

% Niveau 5 (#####)
\titleformat{\subparagraph}{\normalfont\bfseries\small\color{gray!70!black}}{\thesubparagraph}{1em}{}
\titlespacing*{\subparagraph}{0pt}{0.5em plus .1em minus .1em}{0.3em}
\usepackage{etoolbox}     % utile pour redéfinir \tableofcontents
\renewcommand{\contentsname}{}   % enlève le titre "Contents"
% -- TOC compact (~1.5x plus petit) --
\AtBeginDocument{
  \addtocontents{toc}{\protect\smallskip}
  \let\oldtableofcontents\tableofcontents
  \renewcommand{\tableofcontents}{
    \begingroup
      \footnotesize                % <- ~1.5x plus petit
      \setlength{\parskip}{2pt}    % espacement entre entrées
      \oldtableofcontents
    \endgroup
  }
}

\setcounter{tocdepth}{5}
\makeatletter
\renewcommand{\@tocrmarg}{0pt}
\makeatother
\usepackage{fvextra}
\usepackage[section]{placeins}
\usepackage{needspace}
\usepackage{float}
\floatplacement{figure}{H}
\floatplacement{table}{H}
\DefineVerbatimEnvironment{Highlighting}{Verbatim}{breaklines,breakanywhere,fontsize=\small,commandchars=\\\{\},samepage=true}
\newcommand{\sectionbreak}{\needspace{5\baselineskip}}
\setlength{\parindent}{0pt}
\setlength{\parskip}{4pt}
\usepackage[most]{tcolorbox}
\newtcolorbox{graybox}{colback=gray!10!white,colframe=black,boxrule=0.6pt,arc=1mm,left=6pt,right=6pt,top=4pt,bottom=4pt}
\newtcolorbox{orangebox}{colback=orange!10!white,colframe=black,boxrule=0.6pt,arc=1mm,left=6pt,right=6pt,top=4pt,bottom=4pt}
\newtcolorbox{codebox}{breakable,colback=blue!5!white,colframe=blue!50!black,boxrule=0.5pt,arc=1mm,left=4pt,right=4pt,top=3pt,bottom=3pt}
\DefineVerbatimEnvironment{CodeBoxContent}{Verbatim}{fontsize=\small,breaklines,breakanywhere}
\renewcommand{\thesection}{\arabic{section}}
\renewcommand{\thesubsection}{\thesection.\Alph{subsection}}
\renewcommand{\thesubsubsection}{\thesubsection.\arabic{subsubsection}}
\usepackage{booktabs}
\usepackage{longtable}
\usepackage{array}
\usepackage{multirow}
\usepackage{wrapfig}
\usepackage{float}
\usepackage{colortbl}
\usepackage{pdflscape}
\usepackage{tabu}
\usepackage{threeparttable}
\usepackage{threeparttablex}
\usepackage[normalem]{ulem}
\usepackage{makecell}
\usepackage{xcolor}
\usepackage{caption}
\usepackage{anyfontsize}
\usepackage{bookmark}
\IfFileExists{xurl.sty}{\usepackage{xurl}}{} % add URL line breaks if available
\urlstyle{same}
\hypersetup{
  pdftitle={Stats Thèse Max},
  pdfauthor={Thomas HUSSON},
  hidelinks,
  pdfcreator={LaTeX via pandoc}}

\title{Stats Thèse Max}
\author{Thomas HUSSON}
\date{2025-10-21}

\begin{document}
\maketitle

{
\setcounter{tocdepth}{5}
\tableofcontents
}
\section{Import de la base de
données}\label{import-de-la-base-de-donnuxe9es}

\begin{Shaded}
\begin{Highlighting}[]
\FunctionTok{gs4\_deauth}\NormalTok{()}
\NormalTok{df }\OtherTok{\textless{}{-}} \FunctionTok{read\_sheet}\NormalTok{(}
  \StringTok{"https://docs.google.com/spreadsheets/d/1eWwPK8G89G6nWTDzWimcCa8EOWqval8RPvwykZmfGoI/edit?gid=803820517\#gid=803820517"}\NormalTok{,}
  \AttributeTok{sheet =} \StringTok{"CopieThomas"}
\NormalTok{)}
\end{Highlighting}
\end{Shaded}

\section{Statistiques descriptives}\label{statistiques-descriptives}

\subsection{Recodages des variables pour
lisibilité}\label{recodages-des-variables-pour-lisibilituxe9}

\begin{Shaded}
\begin{Highlighting}[]
\CommentTok{\#Factor l\textquotesingle{}Âge dans le bon ordre : }
\NormalTok{df }\OtherTok{\textless{}{-}}\NormalTok{ df }\SpecialCharTok{\%\textgreater{}\%}
  \FunctionTok{mutate}\NormalTok{(}\StringTok{\textasciigrave{}}\AttributeTok{3\_Age}\StringTok{\textasciigrave{}} \OtherTok{=} \FunctionTok{factor}\NormalTok{(}\StringTok{\textasciigrave{}}\AttributeTok{3\_Age}\StringTok{\textasciigrave{}}\NormalTok{,}
                          \AttributeTok{levels =} \FunctionTok{c}\NormalTok{(}\StringTok{"Entre 30 et 39 ans"}\NormalTok{,}
                                     \StringTok{"Entre 40 et 49 ans"}\NormalTok{,}
                                     \StringTok{"Entre 50 et 59 ans"}\NormalTok{,}
                                     \StringTok{"Entre 60 et 69 ans"}\NormalTok{,}
                                     \StringTok{"Plus de 70 ans"}\NormalTok{)))}


\CommentTok{\#Factor la durée d\textquotesingle{}installation dans le bon ordre }
\NormalTok{df }\OtherTok{\textless{}{-}}\NormalTok{ df }\SpecialCharTok{\%\textgreater{}\%}
  \FunctionTok{mutate}\NormalTok{(}\StringTok{\textasciigrave{}}\AttributeTok{6\_Duree\_d\_installation}\StringTok{\textasciigrave{}} \OtherTok{=} \FunctionTok{factor}\NormalTok{(}\StringTok{\textasciigrave{}}\AttributeTok{6\_Duree\_d\_installation}\StringTok{\textasciigrave{}}\NormalTok{,}
                                           \AttributeTok{levels =} \FunctionTok{c}\NormalTok{(}\StringTok{"Moins de 5 ans"}\NormalTok{,}
                                                      \StringTok{"Entre 5 et 9 ans"}\NormalTok{,}
                                                      \StringTok{"Entre 10 et 19 ans"}\NormalTok{,}
                                                      \StringTok{"Plus de 20 ans"}\NormalTok{)))}



\CommentTok{\#Factor les types d\textquotesingle{}activité dans le bon ordre }
\NormalTok{df }\OtherTok{\textless{}{-}}\NormalTok{ df }\SpecialCharTok{\%\textgreater{}\%}
  \FunctionTok{mutate}\NormalTok{(}\StringTok{\textasciigrave{}}\AttributeTok{7\_Type\_d\_activite}\StringTok{\textasciigrave{}} \OtherTok{=} \FunctionTok{factor}\NormalTok{(}\StringTok{\textasciigrave{}}\AttributeTok{7\_Type\_d\_activite}\StringTok{\textasciigrave{}}\NormalTok{,}
                                      \AttributeTok{levels =} \FunctionTok{c}\NormalTok{(}\StringTok{"Exclusivement libéral en cabinet"}\NormalTok{,}
                                                 \StringTok{"Essentiellement libéral avec activité universitaire"}\NormalTok{,}
                                                 \StringTok{"Essentiellement libéral avec activité de régulation/PDSA"}\NormalTok{,}
                                                 \StringTok{"Mixte (libéral + hospitalière)"}\NormalTok{,}
                                                 \StringTok{"Autre"}\NormalTok{)))}
\end{Highlighting}
\end{Shaded}

\subsection{Description de chaques médecins : par
graphiques}\label{description-de-chaques-muxe9decins-par-graphiques}

\subsubsection{Connaissance MCS}\label{connaissance-mcs}

\begin{itemize}
\tightlist
\item
  Variable : df\$\texttt{1\_Connaissance\_MCS\_binaire}
\end{itemize}

\begin{center}\includegraphics[width=1\linewidth]{Rapport-Thèse-Max_files/figure-latex/camembert connaissance-1} \end{center}

\subsubsection{Sexe}\label{sexe}

\begin{center}\includegraphics[width=1\linewidth]{Rapport-Thèse-Max_files/figure-latex/camembert sexe-1} \end{center}

\subsubsection{Âge}\label{uxe2ge}

\begin{center}\includegraphics[width=1\linewidth]{Rapport-Thèse-Max_files/figure-latex/unnamed-chunk-5-1} \end{center}

\subsubsection{Lieu d'installation : carte des
répondants}\label{lieu-dinstallation-carte-des-ruxe9pondants}

\subsubsection{Durée d'installation :}\label{duruxe9e-dinstallation}

\begin{center}\includegraphics[width=1\linewidth]{Rapport-Thèse-Max_files/figure-latex/unnamed-chunk-7-1} \end{center}

\subsubsection{Type d'activité :}\label{type-dactivituxe9}

\begin{center}\includegraphics[width=1\linewidth]{Rapport-Thèse-Max_files/figure-latex/unnamed-chunk-8-1} \end{center}

Représentation en camembert :

\begin{center}\includegraphics[width=1\linewidth]{Rapport-Thèse-Max_files/figure-latex/camembert_type_activite-1} \end{center}

\subsubsection{Types de visites}\label{types-de-visites}

\begin{itemize}
\tightlist
\item
  Variables incluses : 8\_\_consultations\_rdv
  8\_\_consultations\_sans\_rdv\_
  8\_consultations\_:\emph{creneaux\_d\_urgence 8\_Visites 8\_Cs\_autre
  8bis\_En\_cas\_de\_reponse}''Autre''\emph{merci\_de\_preciser}{[}Commentaire{]}
\end{itemize}

\begin{center}\includegraphics[width=1\linewidth]{Rapport-Thèse-Max_files/figure-latex/unnamed-chunk-9-1} \end{center}

\subsubsection{Ressenti sur le délai d'intervention du SMUR
:}\label{ressenti-sur-le-duxe9lai-dintervention-du-smur}

\begin{itemize}
\tightlist
\item
  Variable : df\$\texttt{9\_Ressenti\_delai\_SMUR}
\end{itemize}

\begin{center}\includegraphics[width=1\linewidth]{Rapport-Thèse-Max_files/figure-latex/unnamed-chunk-10-1} \end{center}

\begin{itemize}
\item
  \textbf{AUTRES} : représentés par :

  \begin{itemize}
  \item
    ``\emph{Pas ou peu d'urgences vraies. Pour la semi urgence, on a
    toujours réussi à se dépatouiller : amélioration de l'état clinique
    par les médicaments sur place ou récupérés à la pharma en urgence
    par la famille ou alors transport hospitalier ''\,''rapide''\,'' 0
    médicalisé par la famille. Par contre en cas d'urgence réelle, je
    pressens que l'équation pourrait être problématique.}''
  \item
    ``\emph{Il y a un cabinet qui gère les urgences à 50 mètres du
    mien}''
  \item
    ``\emph{Délai de prise en charge fortement modulé selon
    l'utilisation de l'hélicoptère ou 0, intérêt+++ de la télé médecine
    au sein du CH de Cilaos}''
  \item
    ``\emph{les délais sont longs mais la perception dépend de
    l'urgence}''
  \end{itemize}
\end{itemize}

\paragraph{Idem mais camembert avec mêmes
couleurs}\label{idem-mais-camembert-avec-muxeames-couleurs}

\begin{itemize}
\tightlist
\item
  Variable : df\$\texttt{9\_Ressenti\_delai\_SMUR}
\end{itemize}

\begin{center}\includegraphics[width=1\linewidth]{Rapport-Thèse-Max_files/figure-latex/camembert-1} \end{center}

\paragraph{Carte}\label{carte}

\subsubsection{Perte de chance dans le secteur liée au délai
d'intervention du
SMUR}\label{perte-de-chance-dans-le-secteur-liuxe9e-au-duxe9lai-dintervention-du-smur}

\paragraph{Histogramme}\label{histogramme}

\begin{center}\includegraphics[width=1\linewidth]{Rapport-Thèse-Max_files/figure-latex/unnamed-chunk-12-1} \end{center}

\paragraph{Camembert}\label{camembert}

\begin{center}\includegraphics[width=1\linewidth]{Rapport-Thèse-Max_files/figure-latex/unnamed-chunk-13-1} \end{center}

\paragraph{Carte}\label{carte-1}

\paragraph{Tableau récapitulatif}\label{tableau-ruxe9capitulatif}

2 colonnes incluses :

\begin{itemize}
\item
  `9\_Ressenti\_delai\_SMUR``
\item
  `10\_Delai\_d\_intervention\_:\_perte\_de\_chance\_dans\_votre\_secteur``
\end{itemize}

\begin{Shaded}
\begin{Highlighting}[]
\NormalTok{cols\_to\_include2 }\OtherTok{\textless{}{-}} \FunctionTok{c}\NormalTok{(}
  \StringTok{"9\_Ressenti\_delai\_SMUR"}\NormalTok{,}
  \StringTok{"10\_Delai\_d\_intervention\_:\_perte\_de\_chance\_dans\_votre\_secteur"}
\NormalTok{)}
    
\CommentTok{\#Ordonner les facteurs pour lisibilité}
\NormalTok{df }\OtherTok{\textless{}{-}}\NormalTok{ df }\SpecialCharTok{\%\textgreater{}\%}
  \FunctionTok{mutate}\NormalTok{(}\StringTok{\textasciigrave{}}\AttributeTok{9\_Ressenti\_delai\_SMUR}\StringTok{\textasciigrave{}} \OtherTok{=} \FunctionTok{factor}\NormalTok{(}\StringTok{\textasciigrave{}}\AttributeTok{9\_Ressenti\_delai\_SMUR}\StringTok{\textasciigrave{}}\NormalTok{,}
                                          \AttributeTok{levels =} \FunctionTok{c}\NormalTok{(}\StringTok{"Délai ok"}\NormalTok{,}
                                                     \StringTok{"Plutôt bien mais parfois gêné"}\NormalTok{,}
                                                     \StringTok{"Délai trop long, en difficulté"}\NormalTok{,}
                                                     \StringTok{"Autre"}\NormalTok{)))}
\NormalTok{df }\OtherTok{\textless{}{-}}\NormalTok{ df }\SpecialCharTok{\%\textgreater{}\%}
  \FunctionTok{mutate}\NormalTok{(}\StringTok{\textasciigrave{}}\AttributeTok{10\_Delai\_d\_intervention\_:\_perte\_de\_chance\_dans\_votre\_secteur}\StringTok{\textasciigrave{}} \OtherTok{=} \FunctionTok{factor}\NormalTok{(}\StringTok{\textasciigrave{}}\AttributeTok{10\_Delai\_d\_intervention\_:\_perte\_de\_chance\_dans\_votre\_secteur}\StringTok{\textasciigrave{}}\NormalTok{,}
                                                                                   \AttributeTok{levels =} \FunctionTok{c}\NormalTok{(}\StringTok{"non, pas du tout"}\NormalTok{,}\StringTok{"Plutôt non"}\NormalTok{,}\StringTok{"Plutôt oui"}\NormalTok{,}\StringTok{"oui, tout à fait"}\NormalTok{)))}

\NormalTok{table2 }\OtherTok{\textless{}{-}}\NormalTok{ df }\SpecialCharTok{\%\textgreater{}\%} 
  \FunctionTok{tbl\_summary}\NormalTok{(}
    \AttributeTok{include =} \FunctionTok{all\_of}\NormalTok{(cols\_to\_include2), }\CommentTok{\# colonnes à inclure}
    \AttributeTok{statistic =} \FunctionTok{list}\NormalTok{(}
      \FunctionTok{all\_categorical}\NormalTok{() }\SpecialCharTok{\textasciitilde{}} \StringTok{"\{n\} (\{p\}\%)"} \CommentTok{\# n (\%) pour les variables catégorielles}
\NormalTok{    ),}
    \AttributeTok{label =} \FunctionTok{list}\NormalTok{(}
      \StringTok{\textasciigrave{}}\AttributeTok{9\_Ressenti\_delai\_SMUR}\StringTok{\textasciigrave{}} \OtherTok{=} \StringTok{"Ressenti sur le délai d\textquotesingle{}intervention du SMUR"}\NormalTok{,}
      \StringTok{\textasciigrave{}}\AttributeTok{10\_Delai\_d\_intervention\_:\_perte\_de\_chance\_dans\_votre\_secteur}\StringTok{\textasciigrave{}} \OtherTok{=} \StringTok{"Perte de chance dans votre secteur liée au délai d\textquotesingle{}intervention"}
\NormalTok{    ),}
    \AttributeTok{missing =} \StringTok{"no"} \CommentTok{\# ne pas inclure les valeurs manquantes dans le tableau}
\NormalTok{  ) }\SpecialCharTok{\%\textgreater{}\%}
  \FunctionTok{modify\_header}\NormalTok{(}\AttributeTok{label =} \StringTok{"**Caractéristiques**"}\NormalTok{) }\SpecialCharTok{\%\textgreater{}\%} \CommentTok{\# modifier l\textquotesingle{}en{-}tête de la colonne des labels}
  \FunctionTok{bold\_labels}\NormalTok{() }\CommentTok{\# mettre en gras les labels des variables}

\NormalTok{table2}
\end{Highlighting}
\end{Shaded}

\begin{table}[t]
\fontsize{12.0pt}{14.4pt}\selectfont
\begin{tabular*}{\linewidth}{@{\extracolsep{\fill}}lc}
\toprule
\textbf{Caractéristiques} & \textbf{N = 54}\textsuperscript{\textit{1}} \\ 
\midrule\addlinespace[2.5pt]
{\bfseries Ressenti sur le délai d'intervention du SMUR} &  \\ 
    Délai ok & 0 (0\%) \\ 
    Plutôt bien mais parfois gêné & 32 (76\%) \\ 
    Délai trop long, en difficulté & 5 (12\%) \\ 
    Autre & 5 (12\%) \\ 
{\bfseries Perte de chance dans votre secteur liée au délai d'intervention} &  \\ 
    non, pas du tout & 3 (5.6\%) \\ 
    Plutôt non & 18 (33\%) \\ 
    Plutôt oui & 29 (54\%) \\ 
    oui, tout à fait & 4 (7.4\%) \\ 
\bottomrule
\end{tabular*}
\begin{minipage}{\linewidth}
\textsuperscript{\textit{1}}n (\%)\\
\end{minipage}
\end{table}

\paragraph{Perte de chance binaire liée au délai
d'intervention}\label{perte-de-chance-binaire-liuxe9e-au-duxe9lai-dintervention}

\begin{Shaded}
\begin{Highlighting}[]
\NormalTok{slices }\OtherTok{\textless{}{-}} \FunctionTok{table}\NormalTok{(df}\SpecialCharTok{$}\StringTok{\textasciigrave{}}\AttributeTok{10\_Delai\_d\_intervention\_:\_perte\_de\_chance\_dans\_votre\_secteur\_binaire}\StringTok{\textasciigrave{}}\NormalTok{)}
\NormalTok{labels }\OtherTok{\textless{}{-}} \FunctionTok{c}\NormalTok{(}\StringTok{"Non"}\NormalTok{, }\StringTok{"Oui"}\NormalTok{)}
\NormalTok{pct }\OtherTok{\textless{}{-}} \FunctionTok{round}\NormalTok{(slices }\SpecialCharTok{/} \FunctionTok{sum}\NormalTok{(slices) }\SpecialCharTok{*} \DecValTok{100}\NormalTok{)}
\NormalTok{labels }\OtherTok{\textless{}{-}} \FunctionTok{paste}\NormalTok{(labels, pct) }\CommentTok{\# Ajoute les pourcentages aux labels}
\NormalTok{labels }\OtherTok{\textless{}{-}} \FunctionTok{paste}\NormalTok{(labels, }\StringTok{"\%"}\NormalTok{, }\AttributeTok{sep =} \StringTok{""}\NormalTok{) }\CommentTok{\# Ajoute le symbole \%}
\FunctionTok{pie}\NormalTok{(slices,}
    \AttributeTok{main =} \StringTok{"Perte de chance binaire liée au délai d\textquotesingle{}intervention"}\NormalTok{,}
    \AttributeTok{col =} \FunctionTok{c}\NormalTok{(}\StringTok{"lightcoral"}\NormalTok{, }\StringTok{"lightgreen"}\NormalTok{),}
    \AttributeTok{labels =}\NormalTok{ labels}
\NormalTok{)}
\end{Highlighting}
\end{Shaded}

\begin{center}\includegraphics[width=1\linewidth]{Rapport-Thèse-Max_files/figure-latex/unnamed-chunk-16-1} \end{center}

\subsubsection{Réseau MCS est-il pertinent pour la Réunion
?}\label{ruxe9seau-mcs-est-il-pertinent-pour-la-ruxe9union}

\paragraph{Histogramme}\label{histogramme-1}

\begin{itemize}
\tightlist
\item
  Variable : df\$\texttt{11\_Reseau\_MCS\_pertinent\_pour\_La\_Reunion}
\end{itemize}

\begin{center}\includegraphics[width=1\linewidth]{Rapport-Thèse-Max_files/figure-latex/unnamed-chunk-18-1} \end{center}

\paragraph{Camembert}\label{camembert-1}

\begin{itemize}
\tightlist
\item
  Variable : df\$\texttt{11\_Reseau\_MCS\_pertinent\_pour\_La\_Reunion}
\end{itemize}

\begin{center}\includegraphics[width=1\linewidth]{Rapport-Thèse-Max_files/figure-latex/camembert pertinent-1} \end{center}

\subsubsection{Délai depuis dernière formation aux soins
d'urgences}\label{duxe9lai-depuis-derniuxe8re-formation-aux-soins-durgences}

\begin{itemize}
\tightlist
\item
  Variable df\$\texttt{12\_Dernieres\_formations\_d\_urgence}
\end{itemize}

\begin{center}\includegraphics[width=1\linewidth]{Rapport-Thèse-Max_files/figure-latex/unnamed-chunk-19-1} \end{center}

\textbf{Visualisation du rapport entre l'Âge et le délai depuis la
dernière formation aux soins d'urgence}

\begin{center}\includegraphics[width=1\linewidth]{Rapport-Thèse-Max_files/figure-latex/unnamed-chunk-20-1} \end{center}

\hfill\break
\hfill\break

\textbf{Autres visualisations pour monsieur}

\begin{center}\includegraphics[width=1\linewidth]{Rapport-Thèse-Max_files/figure-latex/unnamed-chunk-21-1} \end{center}

\hfill\break
\hfill\break
\#\#\# Cabinet adapté aux urgences

\begin{itemize}
\tightlist
\item
  Variable df\$\texttt{13\_Cabinet\_adapte\_aux\_urgences}
\end{itemize}

\paragraph{Histogramme}\label{histogramme-2}

\begin{center}\includegraphics[width=1\linewidth]{Rapport-Thèse-Max_files/figure-latex/unnamed-chunk-22-1} \end{center}

\paragraph{Camembert}\label{camembert-2}

\begin{center}\includegraphics[width=1\linewidth]{Rapport-Thèse-Max_files/figure-latex/camembert cabinet-1} \end{center}

\hfill\break
\hfill\break
\hfill\break
\hfill\break
\#\#\# Intérêt pour une formation complémentaire en urgence

\begin{itemize}
\tightlist
\item
  Variable :
  df\$\texttt{14\_Interet\_pour\_formation\_complementaire\_en\_urgence}
\end{itemize}

\paragraph{Histogramme}\label{histogramme-3}

\begin{center}\includegraphics[width=1\linewidth]{Rapport-Thèse-Max_files/figure-latex/unnamed-chunk-23-1} \end{center}

\paragraph{Camembert}\label{camembert-3}

\begin{center}\includegraphics[width=1\linewidth]{Rapport-Thèse-Max_files/figure-latex/camembert formation urgence-1} \end{center}

\hfill\break
\hfill\break
\hfill\break
\hfill\break
\#\#\# Formation incite à être MCS

\begin{itemize}
\tightlist
\item
  Variable :
  df\$\texttt{15\_Si\_+\_forme\_aux\_urgences:\_incitation\_à\_devenir\_MCS}
\end{itemize}

\paragraph{Histogramme}\label{histogramme-4}

\begin{center}\includegraphics[width=1\linewidth]{Rapport-Thèse-Max_files/figure-latex/unnamed-chunk-24-1} \end{center}

\paragraph{Camembert}\label{camembert-4}

\begin{center}\includegraphics[width=1\linewidth]{Rapport-Thèse-Max_files/figure-latex/camembert incitation MCS-1} \end{center}

\hfill\break
\hfill\break
\hfill\break
\hfill\break

\subsubsection{Matériel incite à être
MCS}\label{matuxe9riel-incite-uxe0-uxeatre-mcs}

\begin{itemize}
\tightlist
\item
  Variable :
  df\$\texttt{16\_Materiel\_adapte\_à\_l’urgence\_:\_incitation\_à\_devenir\_MCS}
\end{itemize}

\paragraph{Histogramme}\label{histogramme-5}

\begin{center}\includegraphics[width=1\linewidth]{Rapport-Thèse-Max_files/figure-latex/unnamed-chunk-25-1} \end{center}

\paragraph{Camembert}\label{camembert-5}

\begin{center}\includegraphics[width=1\linewidth]{Rapport-Thèse-Max_files/figure-latex/camembert incitation materiel-1} \end{center}

\hfill\break
\hfill\break
\#\# Motivations et freins

\subsubsection{Motivations à devenir
MCS}\label{motivations-uxe0-devenir-mcs}

\begin{center}\includegraphics[width=1\linewidth]{Rapport-Thèse-Max_files/figure-latex/motivations-1} \end{center}

\begin{center}\includegraphics[width=1\linewidth]{Rapport-Thèse-Max_files/figure-latex/motivations2-1} \end{center}

\hfill\break
\hfill\break
\#\#\# Freins

\begin{itemize}
\item
  Variables :

  \begin{itemize}
  \item
    df\$\texttt{18\_Freins\_{[}Charge\_de\_travail\_supplementaire{]}}
  \item
    df\$\texttt{18\_Freins\_{[}Manque\_de\_formation\_en\_urgence{]}}
  \item
    df\$\texttt{18\_Freins\_{[}Contraintes\_administratives\_ou\_organisationnelles{]}}
  \end{itemize}
\end{itemize}

library(tidyverse)

\begin{center}\includegraphics[width=1\linewidth]{Rapport-Thèse-Max_files/figure-latex/freins-1} \end{center}

\begin{center}\includegraphics[width=1\linewidth]{Rapport-Thèse-Max_files/figure-latex/freins-2} \end{center}

\hfill\break
\hfill\break
\#\# Tableau 1 : Caractéristiques démographiques et professionnelles des
médecins généralistes selon le mode d'exercice

\begin{table}[t]
\fontsize{12.0pt}{14.4pt}\selectfont
\begin{tabular*}{\linewidth}{@{\extracolsep{\fill}}lc}
\toprule
\textbf{Caractéristiques} & \textbf{N = 54}\textsuperscript{\textit{1}} \\ 
\midrule\addlinespace[2.5pt]
{\bfseries Connaissance du réseau MCS} & 4 (7.4\%) \\ 
{\bfseries Sexe (Homme)} & 29 (54\%) \\ 
{\bfseries Âge} &  \\ 
    Entre 30 et 39 ans & 17 (31\%) \\ 
    Entre 40 et 49 ans & 17 (31\%) \\ 
    Entre 50 et 59 ans & 6 (11\%) \\ 
    Entre 60 et 69 ans & 12 (22\%) \\ 
    Plus de 70 ans & 2 (3.7\%) \\ 
{\bfseries Durée d'installation (années)} &  \\ 
    Moins de 5 ans & 21 (39\%) \\ 
    Entre 5 et 9 ans & 6 (11\%) \\ 
    Entre 10 et 19 ans & 11 (20\%) \\ 
    Plus de 20 ans & 16 (30\%) \\ 
{\bfseries Type d'activité} &  \\ 
    Exclusivement libéral en cabinet & 44 (81\%) \\ 
    Essentiellement libéral avec activité universitaire & 3 (5.6\%) \\ 
    Essentiellement libéral avec activité de régulation/PDSA & 2 (3.7\%) \\ 
    Mixte (libéral + hospitalière) & 2 (3.7\%) \\ 
    Autre & 3 (5.6\%) \\ 
{\bfseries Consultations avec rendez-vous} & 25 (46\%) \\ 
{\bfseries Consultations sans rendez-vous} & 48 (89\%) \\ 
{\bfseries Visites} & 39 (72\%) \\ 
{\bfseries Autres consultations} & 2 (3.7\%) \\ 
\bottomrule
\end{tabular*}
\begin{minipage}{\linewidth}
\textsuperscript{\textit{1}}n (\%)\\
\end{minipage}
\end{table}

\section{Comparaison sur CJP : différences entre interessé et non
interessé}\label{comparaison-sur-cjp-diffuxe9rences-entre-interessuxe9-et-non-interessuxe9}

\begin{itemize}
\item
  CJP : recodé en \texttt{df\$CJP} codé ``1''
\item
  Utiliser le recodage binaire pour interprétation plus facile :

  \begin{itemize}
  \item
    connaissance \textless- df\$\texttt{1\_Connaissance\_MCS\_binaire}
  \item
    age \textless- df\$`3\_Age\_inf\_50a
  \item
    sexe \textless- df\$\texttt{2\_Sexe\_Homme}
  \item
    profession\_isolee \textless- df\$\texttt{4\_Profession\_isolee}
  \item
    duree\_installation \textless-
    df\$\texttt{6\_Duree\_d\_installation\_inf\_10ans}
  \item
    activite\_autre \textless-
    df\$\texttt{7\_Activite\_autre\_que\_liberal\_exclusif}
  \item
    ressenti\_delai \textless-
    df\$\texttt{9\_Ressenti\_delai\_SMUR\_genee\_YN}
  \item
    perte\_chance \textless-
    df\$\texttt{10\_Delai\_d\_intervention\_:\_perte\_de\_chance\_dans\_votre\_secteur\_binaire}
  \end{itemize}
\end{itemize}

\begin{table}[t]
\fontsize{12.0pt}{14.4pt}\selectfont
\begin{tabular*}{\linewidth}{@{\extracolsep{\fill}}lccc}
\toprule
\textbf{Caractéristiques} & \textbf{Intéressé par MCS}  N = 24\textsuperscript{\textit{1}} & \textbf{Non intéressé par MCS}  N = 30\textsuperscript{\textit{1}} & \textbf{p-value}\textsuperscript{\textit{2}} \\ 
\midrule\addlinespace[2.5pt]
{\bfseries Connaissance du dispositif MCS} & 3 (13\%) & 1 (3.3\%) & 0.3 \\ 
{\bfseries Âge inférieur à 50 ans} & 17 (71\%) & 17 (57\%) & 0.3 \\ 
{\bfseries Sexe (Homme)} & 15 (63\%) & 14 (47\%) & 0.2 \\ 
{\bfseries Durée d'installation inférieure à 10 ans} & 12 (50\%) & 15 (50\%) & >0.9 \\ 
{\bfseries Activité autre que libéral exclusif} & 6 (25\%) & 4 (13\%) & 0.3 \\ 
{\bfseries Consultations avec rendez-vous} & 8 (33\%) & 17 (57\%) & 0.088 \\ 
{\bfseries Consultations sans rendez-vous} & 21 (88\%) & 27 (90\%) & >0.9 \\ 
{\bfseries Visites} & 19 (79\%) & 20 (67\%) & 0.3 \\ 
{\bfseries Autres consultations} & 2 (8.3\%) & 0 (0\%) & 0.2 \\ 
{\bfseries Ressenti du délai d'intervention du SMUR gêné} & 20 (83\%) & 17 (57\%) & 0.036 \\ 
{\bfseries Perte de chance liée au délai d'intervention (binaire)} & 18 (75\%) & 15 (50\%) & 0.061 \\ 
{\bfseries Dernière formation aux soins d'urgence < 5 ans (binaire)} & 11 (46\%) & 10 (33\%) & 0.3 \\ 
{\bfseries Cabinet adapté aux urgences (binaire)} & 14 (58\%) & 10 (33\%) & 0.066 \\ 
{\bfseries Intérêt pour une formation complémentaire en urgence (binaire)} & 24 (100\%) & 17 (57\%) & <0.001 \\ 
\bottomrule
\end{tabular*}
\begin{minipage}{\linewidth}
\textsuperscript{\textit{1}}n (\%)\\
\textsuperscript{\textit{2}}Fisher's exact test; Pearson's Chi-squared test\\
\end{minipage}
\end{table}

\begin{itemize}
\item
  Je trouve ce tableau interessant !
\item
  Les médecins interssés par être MCS sont significativement :

  \begin{itemize}
  \item
    Plus gêné par le délai d'intervention du SMUR (p \textless{} 0.029)
  \item
    Ont un cabinet + adapté aux urgences p \textless{} 0.05)
  \item
    Sont + intéressés par une formation complémentaire en urgence (p
    \textless{} 0.001)
  \item
    Et d'autres trucs mais regarde
  \end{itemize}
\end{itemize}

Ce tableau est top mais ne permet pas vraiment de \textbf{QUANTIFIER} à
quel point ces facteurs sont associés à l'intérêt pour devenir MCS.

\section{Analyse univariée sur facteurs associés à l'intérêt pour
devenir MCS
(CJP)}\label{analyse-univariuxe9e-sur-facteurs-associuxe9s-uxe0-lintuxe9ruxeat-pour-devenir-mcs-cjp}

\begin{itemize}
\item
  Principe de l'analyse univariée : on compare les caractéristiques des
  médecins selon leur intérêt (oui/non) pour devenir médecin
  correspondant du SAMU dans le cadre du dispositif MCS.
\item
  Variable d'intérêt : \texttt{df\$CJP} (intérêt pour devenir MCS)
\item
  Interprétation :de l'analyse :

  \begin{itemize}
  \item
    Pour chaque variable, on compare la répartition entre les médecins
    intéressés et non intéressés.
  \item
    Les p-values indiquent si les différences observées sont
    statistiquement significatives.
  \item
    Cela permet d'identifier les facteurs potentiellement associés à
    l'intérêt pour devenir MCS.
  \end{itemize}
\item
  Différence avec une multivariée :

  \begin{itemize}
  \item
    L'analyse univariée examine chaque variable indépendamment, tandis
    que la multivariée ajuste pour plusieurs variables simultanément.
  \item
    La multivariée permet d'identifier les facteurs indépendamment
    associés à l'intérêt pour devenir MCS, en tenant compte des
    interactions entre variables.
  \item
    Les résultats peuvent différer entre les deux analyses en raison de
    la prise en compte des confounders dans la multivariée.
  \item
    L'univariée est souvent une étape préliminaire avant la multivariée
    pour sélectionner les variables à inclure.
  \item
    En résumé, l'univariée identifie des associations potentielles,
    tandis que la multivariée évalue les associations indépendantes.
  \end{itemize}
\end{itemize}

\subsection{Analyse multivariée : régression
logistique}\label{analyse-multivariuxe9e-ruxe9gression-logistique}

\subsubsection{Méthode}\label{muxe9thode}

\begin{itemize}
\item
  Choix des variables à inclure :

  \begin{itemize}
  \item
    Nombre limité de variables à inclure car peu de données
  \item
    Donc se limiter à

    \begin{itemize}
    \item
      Peu de variables inclure (\emph{en théorie 1 variable par
      évènement CJP codé ``Oui'' : donc dans ce cas entre 2 et 3
      variables maximum car 25 médecins interessés par être MCS})
    \item
      Seulement les variables qui ont montré une association
      significative ou presque en univarié (pas d'intérêt à inclure des
      variables qui n'ont pas d'effet propre)
    \end{itemize}
  \item
    Méthode utilisé : régression logistique avec correction Firth (évite
    les problèmes de séparation complète ou quasi complète dans les
    modèles de régression logistique, surtout avec de petits
    échantillons ou des événements rares)
  \end{itemize}
\item
  Interprétation :

  \begin{itemize}
  \item
    Odds Ratio (OR) : mesure de l'association entre une variable
    indépendante et la probabilité d'être intéressé pour devenir MCS.
  \item
    Intervalle de confiance (IC) à 95\% : fournie une estimation de la
    précision de l'OR.
  \item
    p-value : indique si l'association est statistiquement significative
    (généralement p \textless{} 0.05).
  \item
    Un OR \textgreater{} 1 indique une association positive, tandis
    qu'un OR \textless{} 1 indique une association négative.

    \begin{itemize}
    \tightlist
    \item
      Exemple : un OR de 2 signifie que les médecins avec cette
      caractéristique ont deux fois plus de chances d'être intéressés
      pour devenir MCS par rapport à ceux sans cette caractéristique.
    \end{itemize}
  \item
    Mutltivariée : on ajuste pour les autres variables incluses dans le
    modèle, permettant d'isoler l'effet propre de chaque variable sur
    l'intérêt pour devenir MCS.
  \end{itemize}
\end{itemize}

logistf(formula =
df\(`19_Apres_toutes_ces_informations,_si_le_dispositif_etait_lance_à_la_Reunion,_seriez-vous_interesses_pour_vous_former_et_devenir_medecin_correspondant_du_SAMU` ~
    df\)\texttt{9\_Ressenti\_delai\_SMUR\_genee\_YN} +
df\(`10_Delai_d_intervention_:_perte_de_chance_dans_votre_secteur_binaire` +
        df\)\texttt{13\_Cabinet\_adapte\_aux\_urgences\_binaire} +
df\$\texttt{14\_Interêt\_pour\_formation\_complementaire\_en\_urgence\_binaire},
data = df)

Model fitted by Penalized ML Coefficients: coef (Intercept) -5.0313406
df\(`9_Ressenti_delai_SMUR_genee_YN`                                        0.8755773
df\)\texttt{10\_Delai\_d\_intervention\_:\_perte\_de\_chance\_dans\_votre\_secteur\_binaire}
0.9527701
df\(`13_Cabinet_adapte_aux_urgences_binaire`                                1.3618301
df\)\texttt{14\_Interêt\_pour\_formation\_complementaire\_en\_urgence\_binaire}
3.5201515 se(coef) (Intercept) 1.6199616
df\(`9_Ressenti_delai_SMUR_genee_YN`                                       0.7425533
df\)\texttt{10\_Delai\_d\_intervention\_:\_perte\_de\_chance\_dans\_votre\_secteur\_binaire}
0.7313179
df\(`13_Cabinet_adapte_aux_urgences_binaire`                               0.6911784
df\)\texttt{14\_Interêt\_pour\_formation\_complementaire\_en\_urgence\_binaire}
1.4115965 lower 0.95 (Intercept) -10.14562068
df\(`9_Ressenti_delai_SMUR_genee_YN`                                        -0.61822751
df\)\texttt{10\_Delai\_d\_intervention\_:\_perte\_de\_chance\_dans\_votre\_secteur\_binaire}
-0.51173187
df\(`13_Cabinet_adapte_aux_urgences_binaire`                                 0.03509346
df\)\texttt{14\_Interêt\_pour\_formation\_complementaire\_en\_urgence\_binaire}
1.31360055 upper 0.95 (Intercept) -2.376497
df\(`9_Ressenti_delai_SMUR_genee_YN`                                         2.460271
df\)\texttt{10\_Delai\_d\_intervention\_:\_perte\_de\_chance\_dans\_votre\_secteur\_binaire}
2.542818
df\(`13_Cabinet_adapte_aux_urgences_binaire`                                 2.891514
df\)\texttt{14\_Interêt\_pour\_formation\_complementaire\_en\_urgence\_binaire}
8.421345 Chisq (Intercept) 21.501326
df\(`9_Ressenti_delai_SMUR_genee_YN`                                        1.322511
df\)\texttt{10\_Delai\_d\_intervention\_:\_perte\_de\_chance\_dans\_votre\_secteur\_binaire}
1.625726
df\(`13_Cabinet_adapte_aux_urgences_binaire`                                4.055035
df\)\texttt{14\_Interêt\_pour\_formation\_complementaire\_en\_urgence\_binaire}
12.882478 p (Intercept) 3.535841e-06
df\(`9_Ressenti_delai_SMUR_genee_YN`                                       2.501419e-01
df\)\texttt{10\_Delai\_d\_intervention\_:\_perte\_de\_chance\_dans\_votre\_secteur\_binaire}
2.022952e-01
df\(`13_Cabinet_adapte_aux_urgences_binaire`                               4.403981e-02
df\)\texttt{14\_Interêt\_pour\_formation\_complementaire\_en\_urgence\_binaire}
3.316727e-04 method (Intercept) 2
df\(`9_Ressenti_delai_SMUR_genee_YN`                                            2
df\)\texttt{10\_Delai\_d\_intervention\_:\_perte\_de\_chance\_dans\_votre\_secteur\_binaire}
2
df\(`13_Cabinet_adapte_aux_urgences_binaire`                                    2
df\)\texttt{14\_Interêt\_pour\_formation\_complementaire\_en\_urgence\_binaire}
2

Method: 1-Wald, 2-Profile penalized log-likelihood, 3-None

Likelihood ratio test=22.37358 on 4 df, p=0.0001688604, n=54 Wald test =
11.32837 on 4 df, p = 0.02311124 Variable (Intercept) (Intercept)
df\(`9_Ressenti_delai_SMUR_genee_YN`                                                                             df\)\texttt{9\_Ressenti\_delai\_SMUR\_genee\_YN}
df\(`10_Delai_d_intervention_:_perte_de_chance_dans_votre_secteur_binaire` df\)\texttt{10\_Delai\_d\_intervention\_:\_perte\_de\_chance\_dans\_votre\_secteur\_binaire}
df\(`13_Cabinet_adapte_aux_urgences_binaire`                                                             df\)\texttt{13\_Cabinet\_adapte\_aux\_urgences\_binaire}
df\(`14_Interêt_pour_formation_complementaire_en_urgence_binaire`                   df\)\texttt{14\_Interêt\_pour\_formation\_complementaire\_en\_urgence\_binaire}
OR (Intercept) 0.006530051
df\(`9_Ressenti_delai_SMUR_genee_YN`                                        2.400260630
df\)\texttt{10\_Delai\_d\_intervention\_:\_perte\_de\_chance\_dans\_votre\_secteur\_binaire}
2.592882341
df\(`13_Cabinet_adapte_aux_urgences_binaire`                                3.903330446
df\)\texttt{14\_Interêt\_pour\_formation\_complementaire\_en\_urgence\_binaire}
33.789546525 CI\_lower (Intercept) 3.924758e-05
df\(`9_Ressenti_delai_SMUR_genee_YN`                                       5.388988e-01
df\)\texttt{10\_Delai\_d\_intervention\_:\_perte\_de\_chance\_dans\_votre\_secteur\_binaire}
5.994565e-01
df\(`13_Cabinet_adapte_aux_urgences_binaire`                               1.035717e+00
df\)\texttt{14\_Interêt\_pour\_formation\_complementaire\_en\_urgence\_binaire}
3.719542e+00 CI\_upper (Intercept) 9.287538e-02
df\(`9_Ressenti_delai_SMUR_genee_YN`                                       1.170798e+01
df\)\texttt{10\_Delai\_d\_intervention\_:\_perte\_de\_chance\_dans\_votre\_secteur\_binaire}
1.271546e+01
df\(`13_Cabinet_adapte_aux_urgences_binaire`                               1.802056e+01
df\)\texttt{14\_Interêt\_pour\_formation\_complementaire\_en\_urgence\_binaire}
4.543009e+03 p\_value (Intercept) 3.535841e-06
df\(`9_Ressenti_delai_SMUR_genee_YN`                                       2.501419e-01
df\)\texttt{10\_Delai\_d\_intervention\_:\_perte\_de\_chance\_dans\_votre\_secteur\_binaire}
2.022952e-01
df\(`13_Cabinet_adapte_aux_urgences_binaire`                               4.403981e-02
df\)\texttt{14\_Interêt\_pour\_formation\_complementaire\_en\_urgence\_binaire}
3.316727e-04

\begin{table}[t]
\caption*{
{\large \textbf{Analyse multivariée par régression logistique}} \\ 
{\small Variables associées à l'intérêt pour devenir MCS}
} 
\fontsize{9.8pt}{11.7pt}\selectfont
\begin{tabular*}{\linewidth}{@{\extracolsep{\fill}}lrlr}
\toprule
Variable & Odds Ratio & 95\% Confidence Interval & p-value \\ 
\midrule\addlinespace[2.5pt]
Gêne liée au délai du SMUR & 2.40 & 0.54 – 11.71 & 0.250 \\ 
Perte de chance perçue dans le secteur & 2.59 & 0.6 – 12.72 & 0.202 \\ 
Cabinet adapté aux urgences & 3.90 & 1.04 – 18.02 & 0.044 \\ 
Intérêt pour formation complémentaire en urgence & 6.12 & 2.88 – 21.1 & 0.002 \\ 
\bottomrule
\end{tabular*}
\end{table}

Variable Odds Ratio 95\% Confidence Interval p-value Gêne liée au délai
du SMUR 2.40 0.54 -- 11.71 0.250 Perte de chance perçue dans le secteur
2.59 0.6 -- 12.72 0.202 Cabinet adapté aux urgences 3.90 1.04 -- 18.02
0.044 Intérêt pour formation complémentaire en urgence 6.12 2.88 -- 21.1
0.002

\subsubsection{Interprétation :}\label{interpruxe9tation}

Variables \textbf{significativement} et \textbf{indépendamment}
associées à l'intérêt pour devenir MCS :

\begin{itemize}
\item
  \textbf{Cabinet adapté aux urgences (OR = 3.90, p = 0.044)} : les
  médecins ayant un cabinet adapté aux urgences ont environ 3.9 fois
  plus de chances d'être intéressés pour devenir MCS par rapport à ceux
  dont le cabinet n'est pas adapté, en ajustant pour les autres
  variables du modèle.
\item
  \textbf{Intérêt pour une formation complémentaire en urgence (OR =
  6.12, p = 0.002)} : les médecins intéressés par une formation
  complémentaire en urgence ont environ 6.12 fois plus de chances d'être
  intéressés pour devenir MCS par rapport à ceux qui ne le sont pas, en
  tenant compte des autres facteurs.
\end{itemize}

Les autres variables (gêne liée au délai du SMUR et perte de chance
perçue dans le secteur) ne sont pas statistiquement significatives dans
ce modèle multivarié, suggérant qu'elles n'ont pas d'effet indépendant
fort sur l'intérêt pour devenir MCS une fois les autres facteurs pris en
compte.

\subsection{Forest plot}\label{forest-plot}

\begin{center}\includegraphics[width=1\linewidth]{Rapport-Thèse-Max_files/figure-latex/forest plot-1} \end{center}

\section{CJP : carte}\label{cjp-carte}

\end{document}
